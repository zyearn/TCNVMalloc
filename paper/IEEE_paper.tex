
%% bare_conf.tex
%% V1.3
%% 2007/01/11
%% by Michael Shell
%% See:
%% http://www.michaelshell.org/
%% for current contact information.
%%
%% This is a skeleton file demonstrating the use of IEEEtran.cls
%% (requires IEEEtran.cls version 1.7 or later) with an IEEE conference paper.
%%
%% Support sites:
%% http://www.michaelshell.org/tex/ieeetran/
%% http://www.ctan.org/tex-archive/macros/latex/contrib/IEEEtran/
%% and
%% http://www.ieee.org/

%%*************************************************************************
%% Legal Notice:
%% This code is offered as-is without any warranty either expressed or
%% implied; without even the implied warranty of MERCHANTABILITY or
%% FITNESS FOR A PARTICULAR PURPOSE! 
%% User assumes all risk.
%% In no event shall IEEE or any contributor to this code be liable for
%% any damages or losses, including, but not limited to, incidental,
%% consequential, or any other damages, resulting from the use or misuse
%% of any information contained here.
%%
%% All comments are the opinions of their respective authors and are not
%% necessarily endorsed by the IEEE.
%%
%% This work is distributed under the LaTeX Project Public License (LPPL)
%% ( http://www.latex-project.org/ ) version 1.3, and may be freely used,
%% distributed and modified. A copy of the LPPL, version 1.3, is included
%% in the base LaTeX documentation of all distributions of LaTeX released
%% 2003/12/01 or later.
%% Retain all contribution notices and credits.
%% ** Modified files should be clearly indicated as such, including  **
%% ** renaming them and changing author support contact information. **
%%
%% File list of work: IEEEtran.cls, IEEEtran_HOWTO.pdf, bare_adv.tex,
%%                    bare_conf.tex, bare_jrnl.tex, bare_jrnl_compsoc.tex
%%*************************************************************************

% *** Authors should verify (and, if needed, correct) their LaTeX system  ***
% *** with the testflow diagnostic prior to trusting their LaTeX platform ***
% *** with production work. IEEE's font choices can trigger bugs that do  ***
% *** not appear when using other class files.                            ***
% The testflow support page is at:
% http://www.michaelshell.org/tex/testflow/



% Note that the a4paper option is mainly intended so that authors in
% countries using A4 can easily print to A4 and see how their papers will
% look in print - the typesetting of the document will not typically be
% affected with changes in paper size (but the bottom and side margins will).
% Use the testflow package mentioned above to verify correct handling of
% both paper sizes by the user's LaTeX system.
%
% Also note that the "draftcls" or "draftclsnofoot", not "draft", option
% should be used if it is desired that the figures are to be displayed in
% draft mode.
%
\documentclass[10pt, conference, compsocconf]{IEEEtran}
\usepackage{graphicx}
\usepackage{balance}  % for  \balance command ON LAST PAGE  (only there!)
\usepackage{amsmath}
\usepackage{algorithm}
\usepackage[noend]{algpseudocode}
\usepackage[T1]{fontenc}
\usepackage[utf8]{inputenc}

% correct bad hyphenation here
% Add the compsocconf option for Computer Society conferences.
%
% If IEEEtran.cls has not been installed into the LaTeX system files,
% manually specify the path to it like:
% \documentclass[conference]{../sty/IEEEtran}





% Some very useful LaTeX packages include:
% (uncomment the ones you want to load)


% *** MISC UTILITY PACKAGES ***
%
%\usepackage{ifpdf}
% Heiko Oberdiek's ifpdf.sty is very useful if you need conditional
% compilation based on whether the output is pdf or dvi.
% usage:
% \ifpdf
%   % pdf code
% \else
%   % dvi code
% \fi
% The latest version of ifpdf.sty can be obtained from:
% http://www.ctan.org/tex-archive/macros/latex/contrib/oberdiek/
% Also, note that IEEEtran.cls V1.7 and later provides a builtin
% \ifCLASSINFOpdf conditional that works the same way.
% When switching from latex to pdflatex and vice-versa, the compiler may
% have to be run twice to clear warning/error messages.






% *** CITATION PACKAGES ***
%
%\usepackage{cite}
% cite.sty was written by Donald Arseneau
% V1.6 and later of IEEEtran pre-defines the format of the cite.sty package
% \cite{} output to follow that of IEEE. Loading the cite package will
% result in citation numbers being automatically sorted and properly
% "compressed/ranged". e.g., [1], [9], [2], [7], [5], [6] without using
% cite.sty will become [1], [2], [5]--[7], [9] using cite.sty. cite.sty's
% \cite will automatically add leading space, if needed. Use cite.sty's
% noadjust option (cite.sty V3.8 and later) if you want to turn this off.
% cite.sty is already installed on most LaTeX systems. Be sure and use
% version 4.0 (2003-05-27) and later if using hyperref.sty. cite.sty does
% not currently provide for hyperlinked citations.
% The latest version can be obtained at:
% http://www.ctan.org/tex-archive/macros/latex/contrib/cite/
% The documentation is contained in the cite.sty file itself.






% *** GRAPHICS RELATED PACKAGES ***
%
\ifCLASSINFOpdf
  % \usepackage[pdftex]{graphicx}
  % declare the path(s) where your graphic files are
  % \graphicspath{{../pdf/}{../jpeg/}}
  % and their extensions so you won't have to specify these with
  % every instance of \includegraphics
  % \DeclareGraphicsExtensions{.pdf,.jpeg,.png}
\else
  % or other class option (dvipsone, dvipdf, if not using dvips). graphicx
  % will default to the driver specified in the system graphics.cfg if no
  % driver is specified.
  % \usepackage[dvips]{graphicx}
  % declare the path(s) where your graphic files are
  % \graphicspath{{../eps/}}
  % and their extensions so you won't have to specify these with
  % every instance of \includegraphics
  % \DeclareGraphicsExtensions{.eps}
\fi
% graphicx was written by David Carlisle and Sebastian Rahtz. It is
% required if you want graphics, photos, etc. graphicx.sty is already
% installed on most LaTeX systems. The latest version and documentation can
% be obtained at: 
% http://www.ctan.org/tex-archive/macros/latex/required/graphics/
% Another good source of documentation is "Using Imported Graphics in
% LaTeX2e" by Keith Reckdahl which can be found as epslatex.ps or
% epslatex.pdf at: http://www.ctan.org/tex-archive/info/
%
% latex, and pdflatex in dvi mode, support graphics in encapsulated
% postscript (.eps) format. pdflatex in pdf mode supports graphics
% in .pdf, .jpeg, .png and .mps (metapost) formats. Users should ensure
% that all non-photo figures use a vector format (.eps, .pdf, .mps) and
% not a bitmapped formats (.jpeg, .png). IEEE frowns on bitmapped formats
% which can result in "jaggedy"/blurry rendering of lines and letters as
% well as large increases in file sizes.
%
% You can find documentation about the pdfTeX application at:
% http://www.tug.org/applications/pdftex





% *** MATH PACKAGES ***
%
%\usepackage[cmex10]{amsmath}
% A popular package from the American Mathematical Society that provides
% many useful and powerful commands for dealing with mathematics. If using
% it, be sure to load this package with the cmex10 option to ensure that
% only type 1 fonts will utilized at all point sizes. Without this option,
% it is possible that some math symbols, particularly those within
% footnotes, will be rendered in bitmap form which will result in a
% document that can not be IEEE Xplore compliant!
%
% Also, note that the amsmath package sets \interdisplaylinepenalty to 10000
% thus preventing page breaks from occurring within multiline equations. Use:
%\interdisplaylinepenalty=2500
% after loading amsmath to restore such page breaks as IEEEtran.cls normally
% does. amsmath.sty is already installed on most LaTeX systems. The latest
% version and documentation can be obtained at:
% http://www.ctan.org/tex-archive/macros/latex/required/amslatex/math/





% *** SPECIALIZED LIST PACKAGES ***
%
%\usepackage{algorithmic}
% algorithmic.sty was written by Peter Williams and Rogerio Brito.
% This package provides an algorithmic environment fo describing algorithms.
% You can use the algorithmic environment in-text or within a figure
% environment to provide for a floating algorithm. Do NOT use the algorithm
% floating environment provided by algorithm.sty (by the same authors) or
% algorithm2e.sty (by Christophe Fiorio) as IEEE does not use dedicated
% algorithm float types and packages that provide these will not provide
% correct IEEE style captions. The latest version and documentation of
% algorithmic.sty can be obtained at:
% http://www.ctan.org/tex-archive/macros/latex/contrib/algorithms/
% There is also a support site at:
% http://algorithms.berlios.de/index.html
% Also of interest may be the (relatively newer and more customizable)
% algorithmicx.sty package by Szasz Janos:
% http://www.ctan.org/tex-archive/macros/latex/contrib/algorithmicx/




% *** ALIGNMENT PACKAGES ***
%
%\usepackage{array}
% Frank Mittelbach's and David Carlisle's array.sty patches and improves
% the standard LaTeX2e array and tabular environments to provide better
% appearance and additional user controls. As the default LaTeX2e table
% generation code is lacking to the point of almost being broken with
% respect to the quality of the end results, all users are strongly
% advised to use an enhanced (at the very least that provided by array.sty)
% set of table tools. array.sty is already installed on most systems. The
% latest version and documentation can be obtained at:
% http://www.ctan.org/tex-archive/macros/latex/required/tools/


%\usepackage{mdwmath}
%\usepackage{mdwtab}
% Also highly recommended is Mark Wooding's extremely powerful MDW tools,
% especially mdwmath.sty and mdwtab.sty which are used to format equations
% and tables, respectively. The MDWtools set is already installed on most
% LaTeX systems. The lastest version and documentation is available at:
% http://www.ctan.org/tex-archive/macros/latex/contrib/mdwtools/


% IEEEtran contains the IEEEeqnarray family of commands that can be used to
% generate multiline equations as well as matrices, tables, etc., of high
% quality.


%\usepackage{eqparbox}
% Also of notable interest is Scott Pakin's eqparbox package for creating
% (automatically sized) equal width boxes - aka "natural width parboxes".
% Available at:
% http://www.ctan.org/tex-archive/macros/latex/contrib/eqparbox/





% *** SUBFIGURE PACKAGES ***
%\usepackage[tight,footnotesize]{subfigure}
% subfigure.sty was written by Steven Douglas Cochran. This package makes it
% easy to put subfigures in your figures. e.g., "Figure 1a and 1b". For IEEE
% work, it is a good idea to load it with the tight package option to reduce
% the amount of white space around the subfigures. subfigure.sty is already
% installed on most LaTeX systems. The latest version and documentation can
% be obtained at:
% http://www.ctan.org/tex-archive/obsolete/macros/latex/contrib/subfigure/
% subfigure.sty has been superceeded by subfig.sty.



%\usepackage[caption=false]{caption}
%\usepackage[font=footnotesize]{subfig}
% subfig.sty, also written by Steven Douglas Cochran, is the modern
% replacement for subfigure.sty. However, subfig.sty requires and
% automatically loads Axel Sommerfeldt's caption.sty which will override
% IEEEtran.cls handling of captions and this will result in nonIEEE style
% figure/table captions. To prevent this problem, be sure and preload
% caption.sty with its "caption=false" package option. This is will preserve
% IEEEtran.cls handing of captions. Version 1.3 (2005/06/28) and later 
% (recommended due to many improvements over 1.2) of subfig.sty supports
% the caption=false option directly:
%\usepackage[caption=false,font=footnotesize]{subfig}
%
% The latest version and documentation can be obtained at:
% http://www.ctan.org/tex-archive/macros/latex/contrib/subfig/
% The latest version and documentation of caption.sty can be obtained at:
% http://www.ctan.org/tex-archive/macros/latex/contrib/caption/




% *** FLOAT PACKAGES ***
%
%\usepackage{fixltx2e}
% fixltx2e, the successor to the earlier fix2col.sty, was written by
% Frank Mittelbach and David Carlisle. This package corrects a few problems
% in the LaTeX2e kernel, the most notable of which is that in current
% LaTeX2e releases, the ordering of single and double column floats is not
% guaranteed to be preserved. Thus, an unpatched LaTeX2e can allow a
% single column figure to be placed prior to an earlier double column
% figure. The latest version and documentation can be found at:
% http://www.ctan.org/tex-archive/macros/latex/base/



%\usepackage{stfloats}
% stfloats.sty was written by Sigitas Tolusis. This package gives LaTeX2e
% the ability to do double column floats at the bottom of the page as well
% as the top. (e.g., "\begin{figure*}[!b]" is not normally possible in
% LaTeX2e). It also provides a command:
%\fnbelowfloat
% to enable the placement of footnotes below bottom floats (the standard
% LaTeX2e kernel puts them above bottom floats). This is an invasive package
% which rewrites many portions of the LaTeX2e float routines. It may not work
% with other packages that modify the LaTeX2e float routines. The latest
% version and documentation can be obtained at:
% http://www.ctan.org/tex-archive/macros/latex/contrib/sttools/
% Documentation is contained in the stfloats.sty comments as well as in the
% presfull.pdf file. Do not use the stfloats baselinefloat ability as IEEE
% does not allow \baselineskip to stretch. Authors submitting work to the
% IEEE should note that IEEE rarely uses double column equations and
% that authors should try to avoid such use. Do not be tempted to use the
% cuted.sty or midfloat.sty packages (also by Sigitas Tolusis) as IEEE does
% not format its papers in such ways.





% *** PDF, URL AND HYPERLINK PACKAGES ***
%
%\usepackage{url}
% url.sty was written by Donald Arseneau. It provides better support for
% handling and breaking URLs. url.sty is already installed on most LaTeX
% systems. The latest version can be obtained at:
% http://www.ctan.org/tex-archive/macros/latex/contrib/misc/
% Read the url.sty source comments for usage information. Basically,
% \url{my_url_here}.





% *** Do not adjust lengths that control margins, column widths, etc. ***
% *** Do not use packages that alter fonts (such as pslatex).         ***
% There should be no need to do such things with IEEEtran.cls V1.6 and later.
% (Unless specifically asked to do so by the journal or conference you plan
% to submit to, of course. )


\hyphenation{op-tical net-works semi-conduc-tor}


\begin{document}
%
% paper title
% can use linebreaks \\ within to get better formatting as desired

\title{Wamalloc: An thread-cache wear-aware allocator for Non-volatile Main Memory}

% author names and affiliations
% use a multiple column layout for up to two different
% affiliations

\author{\IEEEauthorblockN{Authors Name/s per 1st Affiliation (Author)}
\IEEEauthorblockA{line 1 (of Affiliation): dept. name of organization\\
line 2: name of organization, acronyms acceptable\\
line 3: City, Country\\
line 4: Email: name@xyz.com}
\and
\IEEEauthorblockN{Authors Name/s per 2nd Affiliation (Author)}
\IEEEauthorblockA{line 1 (of Affiliation): dept. name of organization\\
line 2: name of organization, acronyms acceptable\\
line 3: City, Country\\
line 4: Email: name@xyz.com}
}

% conference papers do not typically use \thanks and this command
% is locked out in conference mode. If really needed, such as for
% the acknowledgment of grants, issue a \IEEEoverridecommandlockouts
% after \documentclass

% for over three affiliations, or if they all won't fit within the width
% of the page, use this alternative format:
% 
%\author{\IEEEauthorblockN{Michael Shell\IEEEauthorrefmark{1},
%Homer Simpson\IEEEauthorrefmark{2},
%James Kirk\IEEEauthorrefmark{3}, 
%Montgomery Scott\IEEEauthorrefmark{3} and
%Eldon Tyrell\IEEEauthorrefmark{4}}
%\IEEEauthorblockA{\IEEEauthorrefmark{1}School of Electrical and Computer Engineering\\
%Georgia Institute of Technology,
%Atlanta, Georgia 30332--0250\\ Email: see http://www.michaelshell.org/contact.html}
%\IEEEauthorblockA{\IEEEauthorrefmark{2}Twentieth Century Fox, Springfield, USA\\
%Email: homer@thesimpsons.com}
%\IEEEauthorblockA{\IEEEauthorrefmark{3}Starfleet Academy, San Francisco, California 96678-2391\\
%Telephone: (800) 555--1212, Fax: (888) 555--1212}
%\IEEEauthorblockA{\IEEEauthorrefmark{4}Tyrell Inc., 123 Replicant Street, Los Angeles, California 90210--4321}}




% use for special paper notices
%\IEEEspecialpapernotice{(Invited Paper)}




% make the title area
\maketitle


\begin{abstract}
Non-volatile memory(NVM) is a promising DRAM replacement in computer systems due to its attractive characteristics.
However, low endurance problem has limited its practical applications.
In this paper, we propose Wamalloc, an efficient software level NVM memory allocator to extend PCM's lifetime.
An elaborate hybrid wear-leveling policy is proposed in this paper to achieve wear-leveling without hardware overhead.
The evaluations show that under uniform and random workloads,
the wear-leveling of Wamalloc outperforms that of NVMalloc about xx\% and xx\% ,
and the total memory consumption of Wammaloc is xxX and xxX less than NVMalloc, respecetively.
In addition, the allocation performance is far better than the standard glibc malloc and NVMalloc.

\end{abstract}

\begin{IEEEkeywords}
component; formatting; style; styling;

\end{IEEEkeywords}

\section{Introduction}

For decades, dynamic random-access memory(DRAM) has been used as the main memory of computer system.
However, with the great demand of performance and enerygy-constrain application, the defects of DRAM have become a main concern,
for example, the limited density and its high energy consumption. 
In order to address these problems, an emerging new technology named non-volatile memory(NVM) has been proposed, such as Phase-change memory(PCM), spin transfer torque(STT-RAM), and memristor. 
NVM has the advantages of low latency, low power consumption and byte-addressable, which makes it a better alternative when compared  to DRAM.

However, non-volatile memories suffer from limited write endurance. 
Let us take  as example. A typical PCM cell permanently fails after nearly $10^7$ to $10^9$ writes. 
In such circumstance, a poorly designed allocator will break the PCM in a short time. 
In order to tackle this problem and support a software level wear-aware memory allocator, a new memory allocator targeted for NVM platform must be designed and used in nowadays’ applications running on NVM.

%no NVM allocators at the time of writing can provide both an accurate wear-leveling policy and a good allocation performance simultaneously.
In this paper, we propose a wear-aware allocator that provide wear-awareness without degrading the performance of allocator compared with state-of-the art allocator. 
Specificly, Wamalloc concerns three main aspects of allocator:
(1) an accurate hybrid(fine-grained and coarse-grained) wear-leveling policy;
(2) efficiency;
(3) metadata memory write-reduction.
First, we propose a novel wear-leveling policy that are more accurate;
Besides, we use a thread-cache structure that has a better performance compared to other NVM and DRAM allocators.
At last, Wamalloc decouples metadata and data management.
Generally speaking, putting metadata into DRAM will decrease plenty of writes to NVM, which will probably extend the lifetime of PCM.

Our contributions can be summarized as follows:
\begin{enumerate}
    \item We propose the Wamalloc, which
        (1) uses a hybrid wear-leveling policy focusing on an accurate wear-leveling policy;
        (2) has a good performance compared to the state-of-the-art NVM and DRAM allocators;
        (3) decouples metadata and data management to reduce huge small writes.
    \item We implement Wamalloc as an userspace library, which depends on no special requirements on operating system and hardware modifications.
\end{enumerate}

We have implemented a prototype version of Wamalloc and evaluated its performance and some critical aspects against state-of-the-art memory allocators.
Experiment shows that for most cases Wamalloc outperforms prior system in average block allocation frequency, total memory consumption and allocation latency.

%To our knowledge, Wamalloc ...

\section{Background and Assumptions}

In this section, we will briefly introduce the non-volatile memory and
make some assumptions about operating system support for these memories.

\subsection{Non-Volatile Memory}

These new emerging memory technologie hope to fill the gap boundary between the main memory and external storage.
They are fast, persistent and byte addressable, which make them the best candidates for DRAM.
Three typical technologies are phase change memory, memristor and spin transfer torque .

Let us take PCM as an illustration. PCM is a non-volatile memory that exploits the unique behaviour of chalcogenide glass.
It stores bits by heating a nanoscale piece of chalcogenide glass and makes it to change state by injecting current through memory cell.
Specificly, there are two states.
One is called a crystalline state representing state (1), and the other is called an amorphous state representing state (0).

Besides PCM, the other two alternative technologies(memristor and spin transfer torque) are also completing to be candidates for DRAM.
Although These technologies have good characteristics to be the main memory, they suffer from different kinds of weakness, such as endurance and asymmetry of read and write latency. 
For example, memristor may wear-out after $10^8$ writes, while DRAM has an unlimited endurance.
Table~\ref{tab:NVMvsDRAM} presents the different characteristics of different NVM and DRAM.

\begin{table}[h]
    \centering
    \begin{tabular}{|c|c|c|c|} \hline
        & Read & Write & Endurance\\ \hline
        DRAM & 60ns & 60ns & nearly unlimited\\ \hline
        PCM & 50-85ns & 150ns-1us & $10^8$-$10^{12}$\\ \hline
        Memristor & 100ns & 100ns & $10^8$\\ \hline
        STT-RAM & 6ns & 13ns & $10^{15}$\\ \hline
    \end{tabular}
    \caption{characteristic of NVM and DRAM}
\label{tab:NVMvsDRAM}
\end{table}


In the following sections, we will collectively refer to these memories as NVM, 
because our focus is emphasized on an efficient wear-leveling policy based on non-volatile memory,
instead of the performance of different kinds of NVM.

\subsection{NVM in memory system}

Due to the limitations(limited endurance and slow writes) of NVM,
NVM is not likely to replace DRAM as the only main memory in computer system.
We assume that at least for write-intensive applications,
systems will use a combination of DRAM and NVM, using their respective advantages.
The logical view of the hybrid memory system is shown in Figure~\ref{fig:NVMposition}.
THe NVM library layer provides the ability for applications to manipulate NVM memory pages.

Wamalloc is proposed to be in the layer of NVM library, aiming at providing an efficient wear-aware NVM allocator.

\begin{figure}[h]
\centering
\includegraphics[width=1.5in]{NVMposition.png}
\caption{logical view of memory system}
\label{fig:NVMposition}
\end{figure}

\subsection{Operating system support}

We expect that the virtual memory mechanism can map both volatile and non-volatile memory into the process's address space,
which means a page table(e.g.,in x86) or a inverted page table(e.g.,in PowerPC) is involved when a memory address is read or written.
An application can request NVM memory pages from operating system via an extended system call mmap.

In order to restore to the state that a process was previously in when process restart happens, 
operating system must be able to allow a process to remap its original NVM memory pages to its virtual address space.
Thus, operating system should provide a mechanism to retrive NVM memory pages based on some dedicated NVM management systems.
There are several existing systems to handle this problem\cite{coburn2011nv, satyanarayanan1994lightweight, volos2011mnemosyne}.
In this paper, we assume operating system should provide some access control on NVM memory pages, 
which may be similar to that in file system.

\section{Design of Wamalloc}

In this section, we propose Wamalloc, an memory allocator for non-volatile memory with the dedicated design for wear-aware purpose without the descend of allocation performance.

\subsection{Design Principles}

The most important design goal of Wamalloc is wear-leveling. 
Hence, every time an application makes a memory request to gain a writeable address, 
a memory block with the least allocation time is expected to return.
However, a strict policy that returns the exact block with the least allocation time may introduce extra time penalty.
ln the later section we will see a hybrid method to tackle this problem.

Another requirement of Wamalloc is low latency,
otherwise the memory would probably become the bottleneck of the entire application,
which results in allocator to be useless. 
To reduce the allocation latency, a local heap should be used by every thread to reduce the possible lock contention,
and all memory managed by a local heap should be allocated from a global heap.
however, using lock when request/free memory from global heap is inevitable,
as we will see later, we use some specific methods to reduce the use of lock.

\subsection{Overview Structure}

In this section, we will dig deeper into the design architecture of Wamalloc.
The structure of Wamalloc is showed in Figure~\ref{fig:arch}.
Wamalloc maintains a local heap for each thread, and a global heap for all threads.
The reason why we use such a design is from the obvious observation that if a request from a thread can be satisfied by its local heap,
the allocation latency will be very low since there is no lock contention which may degrade the performance.
However, there must be a global heap that manage all the memory that maintained by each thread,
therefore a lock must be involved. One of our implementation focus is emphasized on how to reduce the use of lock to the best of our ability.

\begin{figure}
\centering
\includegraphics[width=3in]{arch.png}
\caption{overall architecture}
\label{fig:arch}
\end{figure}

Each local heap contains several fix-sized memory chunks.
Each memory chunk is further divided into several memory blocks based on its size class.
Each thread maintains an array of in-use memory chunk, an array of list that contains a list of waiting chunk and a free list to hold empty chunk.
When a thread makes a memory request of some certain memory class and there is no corresponding in-use chunk to answer that request,
the chunk of the same size class in waiting list is checked to become an in-use chunk. 
If still no usable memory chunks can be found,
the free list in the local heap or global heap will be visited based on our wear-aware policy,
which will be discussed in the further subsection.

\subsection{Memory Chunk}

Memory chunk is the basic building block that would be transferred between global heap and local heap. 
A memory chunk consists of two parts: a small chunk header and a 64KB chunk body. 
Chunk body is divided into several memory block based on the size class,
obviously, with smaller size class comes more memory blocks. 
Chunk header stores some metadata such as remaining free blocks number, size class, chunk owner and so on.

Since a memory chunk body only contains 64KB memory, all the memory request smaller than 64KB is regarded as small request,
which is directly handled by local heap without additional locking in the global heap.
Small allocations are divided into several predefined class, just like TCmalloc does. 
A memory request is served by a memory chunk with the nearest size class. 
Wamalloc uses two sets of size class: fine-grained(16B, 24B, 32B,..., 256B), coarse-grained(384B, 512B,..., 32768B, 49152B, 65536B). 
For the memory request larger than 64KB, Wamalloc will redirect the request to operating system via system call such as mmap.

% classes of allocatable objects
\subsubsection{Classes of allocatable blocks}
The chunk overview is shown in Figure~\ref{fig:chunk}.
The usable memory blocks in a memory chunk can be classified into two categories:
first-allocated blocks indicated by the free pointer, which have never been allocated since this chunk is assigned to its local heap,
and reusable blocks indicated by the free list, which have been allocated and freed.
All the blocks starting from the free pointer where first-allocated objects start or at the free list storing reusable objects are ready to be allocated to applications. 
How to organize the free list will be discussed in the following wear-leveling policy subsection.

\begin{figure}[h]
\centering
\includegraphics[width=3.2in]{chunk.png}
\caption{chunk structure}
\label{fig:chunk}
\end{figure}

% how to choose chunk size
\subsubsection{How to choose chunk size}
The size of memory chunk is very critical to the performance of Wamalloc. 
If the chunk size is too small, the memory block will be used up very soon and frequently communicating with the global heap will greatly decrease the performance.
If the chunk size is too large, lots of memory blocks may remain clean or unused, which leads to a huge memory waste. 
A working method used by many other allocators is to choose chunks of small size to serve small blocks, and chunks of large size to serve large blocks.
The reason why this method is adopted is that there will always be available usable memory blocks no matter what size class it belongs to.
The drawbacks are also obvious.
Since the memory chunk size is different, it increases the complexity of global heap to manage these chunks.
Consequently this will cause the extra time penalty and the possible memory fragment, which may lead to more physical memory than an application really needs.

Take above discussions into consideration, we use a uniform memory chunk size for all size class. 
In most cases, memory requests are small allocations, which are naturally guaranteed that there are several usable memory blocks. 
For a special case, if a memory request is 64KB, there is only one block in the entire chunk body, so the allocation policy just view this block as the last remaining block 
and return it to the application. 
After that if the application needs a new 64KB memory, a new chunk is allocated to serve this request.
Further, This design gives us several other benefits: 
\begin{enumerate}
    \item Since all memory chunks are of the same size, then an empty chunk can easily be transferred from one size class to another without any particular actions.
The only overhead is to modify the metadata of the chunk, which can be negligible when compared to the non-uniform chunk size schema. 
    \item The uniform memory chunk simplified the design of global heap, which results in a lower time latency when communicating with global heap.
\end{enumerate}

\subsection{Local Heap}

Local heap is maintained by each thread. 
It is set up when a thread makes the first memory request.
Metadatas, such as size class, are stored in local heap. 
Every chunk in the local heap can be classified into five categories: In-use, Waiting, Full, Not-available, Clean.
The logical view of local heap and its relationship with memory chunk is shown in Figure~\ref{fig:heap}.

\begin{figure}[h]
\centering
\includegraphics[width=3.2in]{heap.png}
\caption{heap structure}
\label{fig:heap}
\end{figure}

\textbf{In-use chunk}. 
When an application makes an allocation request, 
Wamalloc will first find the appropriate size class, then use this class number as index to get the currently using memory chunk. 
All small memory requests are served by in-use chunk.

\textbf{Waiting chunk}. 
Chunks in waiting list are called waiting chunks. 
Waiting chunks are the chunks in a partially allocated state. 
When application use up the whole memory blocks in the current chunk, chunk of the same class size is a candidate waiting to be chosen as a new in-use chunk.

\textbf{Full chunk}. 
After several memory requests to a particular size class, a memory chunk may be out of blocks, which is called a full chunk.
No more blocks can be allocated from a chunk in a Full state, 
thus the following request will cause the allocator to get usable chunk from waiting chunks.
When a block is freed, the chunk it belongs to is in a full state and this full chunk will be added to the waiting list.

\textbf{Not-available chunk}.
Consider all the things we discussed so far, a typical life cycle of a memory chunk could be first used as in-use chunk, then waiting chunk, and back and forth repeatedly. 
For the wear-aware purpose, that is not what we expect. 
So we add a wear-count variable to each memory chunk and define an allocation threshold. 
When a chunk is first allocated from global heap, its wear-count variable is set to zero.
Every time a block is allocated from a chunk, its wear-count variable increases by one. 
If the wear-count variable of a chunk reaches the predefined allocation threshold, it is no longer available for allocation, and its state becomes Not-available. 
When all the blocks in a Not-available chunk are freed, this chunk is returned to global heap for further elaborate wear-leveling policy, which will be discussed in the following subsection.

\textbf{Clean chunk}.
After all the blocks in a waiting chunk are freed, the waiting chunk can be used as a clean chunk without a specific size class.
Hence it should be added to a free chunk pool, becoming a candidate for in-use chunk.
The promotion strategy must be carefully designed to achieve a wear-leveling purpose.

Based on what has been discussed so far,
we can draw the state machine of the relationship between different kinds of chunk in Figure~\ref{fig:state}.

\begin{figure}[h]
\centering
\includegraphics[width=3.2in]{state.png}
\caption{The state machine of chunk state}
\label{fig:state}
\end{figure}

\subsection{Global Heap}

Global heap maintains the whole memory chunks obtained from operating system. 
All the chunks of each thread are allocated from global heap. 
The logic view of global heap is shown in Figure~\ref{fig:global}.

\begin{figure}[h]
\centering
\includegraphics[width=3.2in]{global.png}
\caption{Global heap }
\label{fig:global}
\end{figure}

\subsubsection{Classes of allocatable chunks}
From the figure we can conclude that Wamalloc first requests a huge continuous raw memory and divide it into several chunks ready to be allocated to different threads.
All the chunks starting from the free pointer or at the free list are ready to be allocated, 
just like what we have discussed in block allocation, how to allocate these free chunks is the key to wear-aware strategy.
If a chunk is allocated from the free pointer, Wamalloc just returns the pointer and then increases this pointer by the size of memory chunk, 
otherwise a chunk is allocated from the free list, a wear-leveling policy must be applied.

\subsubsection{Method of enlarging global heap memory}
Global heap is enlarged via system call(e.g., mmap) if there is no available memory chunk.
Since global heap is shared by all threads, a lock is certainly required, which means a performance descend when lock contention happens.
Thus there exists two different opinions on how to increase the memory space. 
One possible solution aims at economic usage of virtual memory, for example, increasing the space linearly;
while the other solution aims at reducing the lock contention if possible, for example, increasing the space exponentially. 
In wamaloc, we use the latter solution for two reasons: 
\begin{enumerate}
    \item With the popularization of 64-bit machine, the size of virtual memory far exceeds the real physical memory an application needs.
Hence, it is unimportant that an application may take too many virtual memory. In return we can decrease the potential lock contention by acquiring the lock as less as possible.
    \item Almost all of the modern CPUs support page table mechanism, 
        which means the physical memory is mapped to some virtual address only when an application really needs the physical memory. 
        There is no page mapping at the virtual memory that an application does not touch, indicating that no physical memory is wasted.
\end{enumerate}

\subsubsection{How to reap chunks of terminated threads}
After a thread is terminated, there may be several remaining chunks in thread's local heap. 
We need to handle these memory chunks properly.
A typical method used in general-purpose memory allocator is to push these chunks into a global stack(FILO queue)
for the reason that these chunks are probably just used by the terminated thread and they may be in CPU's L1/L2/L3 cache, which is a cache-friendly option to reuse the chunk already in cache.
However, it is not a possible option for non-volatile memory because wear-leveling must be taken into consideration.
If we continuously serve an application with the hottest memory chunk, memory ceil would be damaged quickly.
More wear-leveling policy will be discussed in the next subsection.

\subsection{Wear-leveling policy in different level}

In this subsection we will discuss the wear-leveling policy in different level of Wamalloc in detail.
The overall purpose is quite simple: all memory blocks are expected to be allocated evenly.

Traditional method uses first-in-first-out(FIFO) allocation policy(such as queue) or LRU\cite{zhou2009durable}\cite{rodriguez2015write}.
Although the advantage of FIFO is simple and fast, it is not an accurate wear-aware policy.
LRU may be better than FIFO in wear-aware accuracy, but the complexity of maintaining a LRU queue is too high which means a performance drop will happen.

NVMalloc\cite{moraru2013consistent} proposes a method to tackle this problem.
It timestamps every available free memory block to make sure every block is allocated at least in some time interval.
NVMalloc implements it using a don't-allocate list and on every allocation and deallocation,
NVMalloc examines the block at the head of this list.
If it has been in this list for at least time $T$(a predefined threshold), it will be removed from the list and marked as available.
Thus, strictly speaking, it is just a modified FIFO and its drawbacks discussed above still exist.
Further more, making a memory block not available for a time $T$ period will consume more memory spaces. 
It is obvious since when there is no available block in the interval of $T$, the allocator need to ask more memory from operating system for new coming requests.

In Wamalloc, explicit wear-leveling policy should be applied in two different places. 
We propose a novel hybrid wear-leveling method to balance the tradeoff between wear-leveling accuracy and performance.

% wear policy
\subsubsection{In block level}
Our target is to allocate memory blocks as evenly as possible.
Thus it is obviously that Wamalloc should allocate memory block starting from the free pointer first
because these memory blocks have never been allocated since this chunk is assigned to its local heap.
After a chunk runs out of its first-allocated blocks, Wamalloc will search the free list of the chunk to check whether any reusable block is available.
When a thread frees a memory block, this block should be added to the free list of its belonging chunk.
Thus, we should apply wear-leveling policy to this free list.
In consideration of allocation latency, no elaborate mechanism is employed for the sake of performance.
We consider free list as a FIFO queue. Queue may not give an accurate allocation, but in return the allocation latency is small.

\subsubsection{In local heap}

When the current in-use chunk is out of blocks, 
a chunk should be chosen as a new in-use chunk based on the wear-leveling policy.
According to the state machine shown in the previous section,
there are two sources that an available usable chunk can come from:
one is from the waiting chunk list, the other is from the free list of clean chunk in local heap.

Since Clean chunks do not belong to any size class, they are more flexible than the chunks in waiting chunk list.
In other words, if we have some clean chunks in the free list of local heap, 
we do not need to interfere with the global heap
because these clean chunks could meet requests of any size class directly when in-use chunk is out of blocks.
Thus waiting chunk list will be checked first to find the chunk that could replace the current in-use chunk.

Now we will discuss how to organize the waiting chunk list and the clean chunk list.
The method used here should take two things into consideration:
performance and wear-leveling.
Due to the fine-grained wear-leveling policy that will be discussed below,
here we put an emphasis on the performance and provide a coarse-grained wear-leveling policy.
Thus we think FIFO is our best choice, which has a $O(1)$ time complexity and a nearly optimal wear-leveling behavior.

\subsubsection{In global heap}
It is a common case that global heap need to reap the memory chunks returned from local heap.
Hence there is also a free chunk list in global heap maintaining all usable chunks. 
Just like the policy adopted in block level, 
when a new thread requests a new memory chunk as its in-use chunk,
Wamalloc will first allocate memory chunk starting from the free pointer; 
if no available chunk is found, Wamalloc will allocate chunk from free chunk list complied with the following wear-leveling policy.

% hybrid method, 异步四叉堆
The best wear-leveling policy should return the least allocated objects so far. 
To the best of our knowledge, the most suitable data structure is a priority queue using allocation time as its key,
and the time complexity of finding the minimum key is $O(1)$, which sounds perfect to our wear-leveling policy.
However, now that it is the best, it is strange that no wear-aware NVM allocators use it.
We think the main reason is that its time complexity of the push operation is too high, which is $log(n)$, if priority queue is implemented by heaps. 
It will cause an impressive latency when the push operation happens frequently.

In Wamalloc, we use two optimizations make the operations of priority queue not to be the bottleneck:
1) A 4-ary heap is used to implement the priority queue since it has a more CPU-friendlier behavior than other implementations.
2) Chunks are not moved to global pool synchronously, instead, the real operation is taken under the ground asynchronously.

Thus we use our optimized priority queue as the wear-leveling policy used in free chunk list without performance drop.
% The tradeoff is that it may consume a slight more memory space, which is in an acceptable range 

There are two cases when a thread needs to give memory chunk back to the global heap: 
1) when a thread is terminated, all its memory chunks should be returned back.
2) As we have discussed above, the wear-leveling policy in block level is FIFO, which is not an accurate method.
However, we can achieve a nearly optimal policy using following strategy: 
we define a threshold value \textbf{WEAR\_LIMIT} and an allocation time variable \textbf{MaxAllocTime} per chunk. 
Everytime a block is allocated from a chunk, its per chunk variable \textbf{MaxAllocTime} increases by one.
If \textbf{MaxAllocTime} exceed the predefined threshold \textbf{WEAR\_LIMIT}, 
the chunk is no longer available for serving new coming request, and state of the chunk becomes to Not-available.
When all memory blocks in a Not-available chunk are freed by a thread, this chunk will be moved to the global heap.
Instead of directly inserting the chunk to the priority queue in the global heap, Wamalloc will do it asynchronously.
Specificly, the chunk will be first inserted into a producer-consumer queue, which is a $O(1)$ operations;
a background thread will periodically check the producer-consumer queue to get a returned chunk and insert it to the priority queue.
Since the producer-consumer queue is the global variable shared by all threads, another lock is needed.
A problem may be raised when lock contention happen.
As the evaluation section shows, it is not quite frequent that Wamalloc inserts a chunk into the producer-consumer queue,
which means the possible lock contention penalty will be amortized so that it won't be the bottleneck of Wamalloc.

\subsection{Data Placement In DRAM And NVM}

Metadata in Wamalloc can be classified into two different categories: data description and space description.
In this section, we will discuss which category metadatas previously mentioned belong to and how to place them in hybrid memory system.
In Wamalloc, metadatas include chunk header, local heap and global heap.
Chunk header belongs to data description category, 
in which it manages the blocks information inside a chunk and some chunk-related data, 
such as the count of remaining free blocks, size class of the current chunk,
chunk owner, chunk state, wear-leveling count of the chunk and so on.
While local heap and global heap belong to space description,
which are used to advise where the allocator can find the appropriate available chunks.

In traditional memory allocators, metadata and data are coupled with each other very tightly.
When an allocation or a deallocation request comes, 
there will be plenty of small metadata writes of changing pointer to chain its previous block and next block 
in order to split large space or merge small continuous spaces into a large free space.
This access pattern will probably decrease the lifetime of NVM because of the frequent writes to metadatas.
However, it is not what we expect in NVM allocator.

In Wamalloc, we keep all metadatas in DRAM except for chunk header. 
Putting Chunk header in NVM is because when a thread frees a memory block back to allocator, 
a memory pointer \textbf{ptr} is passed as argument to Wamalloc and 
its chunk header is easily computed by \textbf{ptr}(e.g., a simple subtraction),
otherwise another mechanism should be introduced to remember the relationship between a memory address and its chunk header,
which will certainly cause an extra time cost.
At the meantime, Wamalloc keeps the writing frequency of chunk header as less as possible. 
%Figure~\ref{fig:globalview} show the global view of Wamalloc.
%\begin{figure}[ht!]
%\centering
%\includegraphics[width=3.2in]{globalview.png}
%\caption{Global view}
%\label{fig:globalview}
%\end{figure}

\subsection{Allocation Algorithm}

Based on the discussion above, we summarize the allocation algorithm in the following pseudo-code.

\begin{algorithm}
\caption{Allocation Algorithm}\label{euclid}
\begin{algorithmic}[1]
\Procedure{Wamalloc}{size}

\State $\textit{sc} \gets \text{sizeToClass(}\textit{size}\text{);}$

\If {$\text{isSmallClass}(sc)$}
    \State $\text{lh} \gets \text{CurrentLocalHeap();}$
    \State $\text{retry:}$
    \State $\text{ch} \gets \text{InUseChunk(}\textit{lh, sc}\text{);}$
    \State $\text{ptr} \gets \text{AllocFromChunk(}\textit{ch}\text{);}$

    \If {$\text{IsEmptyChunk(}\textit{ch}\text{)}$}
        \State $\text{StateOf(}\textit{ch}\text{)} \gets \text{FULL;}$
        \State $\text{FindNewChunkToReplaceInuse(}\textit{lh, sc}\text{);}$
        \If {$\text{IsNULL(}\textit{ptr}\text{)}$}
            \State \textbf{goto} \emph{retry};
        \EndIf
    \EndIf

    \State \text{IncreaseWearCount(}\textit{ch, 1}\text{);}
    \If {$\text{WearCount(}\textit{ch}\text{)} \ge \textit{WEAR\_LIMIT}}$
        \State $\text{StateOf(}\textit{ch}\text{)} \gets \text{NotAvailable;}$
        \State $\text{FindNewChunkToReplaceInuse(}\textit{lh, sc}\text{);}$
    \EndIf

\Else
    \State $\text{ptr} \gets \text{LargeAlloc(}\textit{size}\text{);}$
\EndIf
\Return $\text{ptr;}$

\EndProcedure
\end{algorithmic}
\end{algorithm}

Wamalloc will first compute the size class according to the size parameter, 
then use this value to find the appropriate in-use chunk.
If it is a small size request, allocation will happen directly in the in-use chunk mentioned above.
Three exceptions could occur during an allocation:
1) the first time a thread makes a request on class $n$, no in-use chunk is available in the current local heap;
2) the in-use chunk is out of free blocks;
3) when the wear-leveling variable reaches the predefined threshold.
All these exceptions cause the same action: the current chunk(in the first situation, it is a NULL pointer) should be replaced by another usable chunk.
If it is a large size request, the request will be redirect to operating system.

The critical path in allocation algorithm is short, which guarantees the performance stability.

\subsection{Deallocation Algorithm}

we summarize the deallocation algorithm in Algorithm 2.

\begin{algorithm}
\caption{Deallocation Algorithm}\label{euclid}
\begin{algorithmic}[1]
\Procedure{wafree}{ptr}

\State $\text{ch} \gets \text{ExtractHeader(ptr);}$
\State $\text{lh} \gets \text{CurrentLocalHeap();}$

\If {$\text{IsLarge(ch)} $}
\State \text{FreeLarge(ch)}
\Else
\State \text{FreeSmall(ch)}
\EndIf

\EndProcedure
\end{algorithmic}
\end{algorithm}

If it is a large memory deallocation, the memory will be returned to operating system directly via system call.
Otherwise it is a small memory deallocation, Wamalloc will perform different actions based on the current chunk state.
Specificly, there are four cases:
(1) if the current chunk is in an in-use state, nothing is done;
(2) in a Full state, it will be added to the corresponding waiting list of local heap;
(3) in a Waiting state and the chunk becomes a clean chunk, it will be added to the free list of local heap;
(4) in a Not-available state and the chunk becomes a clean chunk, it will be returned to the free list of global heap.

\section{Evaluation}

In this section, we evaluate Wamalloc in several benchmarks and compare it against the glibc allocator and NVMalloc\cite{moraru2013consistent}.
We analyze wear-leveling as well as physical memory consumption and the performance of Wamalloc under uniform and random allocation workloads.

\subsection{Benchmark Setup}

All evaluations were performed on a linux machine with 8 Intel Xeon 2GHz processors and 8GB DRAM, running the Linux 3.16 kernel.
We use DRAM as NVM proxy. It is important to note that the evaluation software for NVM is not necessary for our evaluations, for the reason that Wamalloc focuses on the wear-leveling of allocation and the allocator performance, rather than the performance of different NVM.
Thus if the conclusion is true performed in DRAM, it is also true performed in NVM.
Wamalloc is implemented in about 800 lines of code in C.
The \textbf{WEAR\_LIMIT} global variable is set to $50000$ in all evaluations.

We decided to evaluate wear-leveling against NVMalloc and allocation performance against NVMalloc, as well as glibc,
which is compared to provide a performance baseline. 
The only three non-volatile allocator implementations publicly available at the time of writing are NVMalloc, nvm\_malloc\cite{schwalbnvm}, pmemalloc.
Nvm\_malloc provided a recovery mechanism and used the Persistent Memory Emulation Platform(PMEP)\cite{dulloor2014system} to emulate the performance of emerging NVM technologies.
Since Wamalloc uses DRAM as a proxy and do not support recovery, we exclude nvm\_malloc for fairness reasons.
Pmemalloc uses a simple first-fit algorithm and thus performs very poorly at runtime costs with the increasement of allocation.
However, NVMalloc focused largely on designing a wear-aware allocation algorithm to increase the lifetime of NVM, 
and our assumption, evaluation setup and the targeted problem are closest to these in NVMalloc, 
thus we will treat NVMalloc as our main contender.

\subsection{wear-leveling}

First, we evaluate the ability of our allocator to avoid NVM wear-out using a simple test program under uniform and random allocation and deallocation operations(50\% each).
In NVMalloc, for every allocation, the program writes the allocated block once, then use Pin\cite{luk2005pin} to record store to memory and compared the total writes of each block.
In this method, the total writes depend on not only the allocator itself but also the behavior of test program,
which we think may not be necessary and even worse, not be accurate in evaluating the wear-leveling ability of memory allocator.
Instead, it is more accurate to evaluate the allocation frequency of each memory block returned by allocator to illustrate the ability of wear-leveling.
\begin{figure}[t]
\centering
\includegraphics[width=3.2in]{eval_1.png}
\caption{uniform allocations(128byte)}
\label{fig:eval_1}
\end{figure}

\begin{figure}[t]
\centering
\includegraphics[width=3.2in]{eval_3.png}
\caption{random allocations between 64 and 512 bytes}
\label{fig:eval_3}
\end{figure}
More specificly, we run test to get the total allocation memory address, and then storing these addresses into a set \textbf{V},
a collection of distinct memory addresses.
For example, if one specific memory address is allocated twice by allocator, the address appears only once in set \textbf{V}.
The more evenly the addresses in \textbf{V} distribute, the better the wear-leveling policy is.
Thus we use (total allocation time)/(sizeof \textbf{V}) as average allocation frequency to metric the wear-leveling ability.


Figure~\ref{fig:eval_1} and Figure~\ref{fig:eval_3} show the result of wear-leveling based on the metric described above
under uniform allocation of 128 bytes and random allocation between 64 and 512 bytes.
Since Wamalloc use a more elaborate and accurate wear-leveling policy than NVMalloc,
it is expected that Wamalloc performs better than NVMalloc.


In Figure~\ref{fig:eval_1}, as the number of malloc increases,
the average allocation frequency per block of NVMalloc is about 2.
While the value of that in Wamalloc is about 1.98, 
indicating Wamalloc performs better than NVMalloc under the workloads of uniform allocation.
In Figure~\ref{fig:eval_3},
Wamalloc also performs better than NVMalloc under the workload of random allocation.
This can be explained by using the 4-ary heap implementing an accurate and elaborate wear-leveling policy.

However, it is meaningless to compare the average allocation frequency without considering the physical memory consumption.
A simple allocator that allocates every memory block just once and never use that memory again 
could make the average allocation frequency per block to be 1, 
while at the meantime it consumes plenty of physical memory, causing it to be a useless allocator.
Thus, next we are also supposed to measure the memory consumption of above evaluations under uniform and random allocation.

Figure~\ref{fig:eval_2} and Figure~\ref{fig:eval_4} show the total memory consumptions under the same workloads as that in Figure~\ref{fig:eval_1} and Figure~\ref{fig:eval_3}.
We can conclude from these results that 
Wamalloc is more economic in physical memory usage than NVMalloc.
Under uniform allocations, NVMalloc consumes nearly twice more memory than Wamalloc.
Under random allocations, the growth rate of memory consumption of NVMalloc and Wamalloc tend to be linear as the allocating time increases. 
However, Wamalloc still consumes less physical memory than NVMalloc by a constant difference.

\begin{figure}[t]
\centering
\includegraphics[width=3.2in]{eval_2.png}
\caption{physical memory consumption under uniform allocations(128byte)}
\label{fig:eval_2}
\end{figure}

\begin{figure}[t]
\centering
\includegraphics[width=3in]{eval_4.png}
\caption{physical memory consumption under random allocations}
\label{fig:eval_4}
\end{figure}

In the above evaluations, we arrive the conclusion that
Wamalloc performs better than NVMalloc not only in the wear-leveling policy,
but also in the physical memory consumptions.

\subsection{allocation performance}

Now, We are going to compare the allocation performance of Wamalloc against different allocators,
which is a crucial metric since it will affect the overall performance of a program.
Specificly, we compare the average allocation time under workloads of different parallel threads against NVMalloc.
We also add the gblic malloc(version 2.21) into comparison to provide a performance baseline.
Unfortunately, NVMalloc does not support multithreading,
so we modify the source code of NVMalloc to make multithreading supported.

First, we evaluate the average allocation time as the number of parallel threads increases
under the workload of uniform(128 bytes) and random(64 and 512 bytes) allocation without freeing any memory block.
Since Wamalloc uses a thread-cache structure and has a very short critical path using locks as few as possible,
the average allocation time should be little related to the number of threads.
These expectations are confirmed by the result of our evaluations shown in Figure~\ref{fig:eval_5} and Figure~\ref{fig:eval_6}.
From the result we can see that Wamalloc outperforms both glibc and NVMalloc in this evaluation.


\begin{figure}[t]
\centering
\includegraphics[width=3.2in]{eval_5.png}
\caption{average allocation time under uniform allocations(128byte)}
\label{fig:eval_5}
\end{figure}

\begin{figure}[t]
\centering
\includegraphics[width=3.2in]{eval_6.png}
\caption{average allocation time under random allocations}
\label{fig:eval_6}
\end{figure}

Next, a second benchmark was run in which the program is extended to free all the memory after allocations, 
the runtime of which was included in the measurement of average allocation time.
Allocating and then freeing is a common scenario in most applications, thus it is more complicated than just allocating the memory in a multithread environment 
for the reason that freeing may cause changes of the state of some chunks or even some movements of chunk from local heap to global heap.
The evaluation result is shown in Figure~\ref{fig:eval_7}.
Malloc in Glibc performs better than itself when freeing is involved, but the performance still falls behind Wamalloc by a small constant gap.


\begin{figure}[t]
\centering
\includegraphics[width=3.2in]{eval_7.png}
\caption{performance comparison when freeing memory after allocations under random allocations}
\label{fig:eval_7}
\end{figure}

\subsection{Evalution Summary}

As demonstrated, 
the wear-leveling ability of Wamalloc performs better than NVMalloc.
In terms of physical memory consumption, Wamalloc also outperforms NVMalloc
due to the smart memory reuse policy.

We also show that Wamalloc's allocation performance in a multithreading environment is not only better than NVMalloc,
but can also outperform glibc malloc, which is the standard system allocator in most systems.
Under the random pressure of allocating and freeing memory simulating the actual program behavior,
Wamalloc also performs better than NVMalloc and glibc malloc.

\section{Related work}

The emerging technology of NVM has gained more and more attentions in research community.
Proposition were made for different levels of computer system, 
such as using NVM as a fast buffer for flash-based storage\cite{greenan2007prims}, 
or as a supplement for DRAM exploiting the benefit of the hybrid memory system.

In the research area of how to extend the endurance of NVM,
plenty of works have been done at different levels.

At hardware level, ping zhou et al.\cite{zhou2009durable} proposed a suite of hierarchical techniques to
prolong the lifetime of PCM-based main memory: 
redundant bit-write removal, row shifting, and segment swapping.
Lei Jiang et al.\cite{jiang2012improving} proposed write truncation (WT) and form switch (FS),
which is a way to avoid fully writing a small set of difficult-to-write cells, to speedup write and read operations in MLC PCM.
Qureshi, Moinuddin K et al.\cite{qureshi2009enhancing} and Shao, Zili et al.\cite{shao2012ptl} proposed adding a translation layer
between the physical address and the logical address in the memory controller.
Ferreira, Alexandre P et al.\cite{ferreira2010increasing} proposed a technique that involved 
writeback minimization with new cache replacement policies, and wrote only the bit actually changed.

At software level, there are also much works proposed to minimize to the memory writes to NVM, 
such as the research work by Hu, Jingtong et al.\cite{hu2013software}.
Early in the time of research on NVM, mnemosyne\cite{volos2011mnemosyne} and NV-heaps\cite{coburn2011nv} provided complete suites to working with NVM, which involed kernel modifications, user mode liraries and so on.
However, the endurance problem is not considered too much.

A more specific NVM allocator is NVMalloc by Moraru et al.\cite{moraru2013consistent},
intended to ba an efficient, general-purpose NVM allocator with a wear-leveling policy using timestamp.
More specificly, NVMalloc curbs the write frequency of memory block no higher than 1/T, which is a predefined threshold.
Further more, NVMalloc decouples metadata and data to reduce metadata write.
The focus of NVMalloc is mainly on wear-aware algorithms to increase the lifetime of NVM.
The evaluation shows Wamalloc outperforms NVMalloc in terms of wear-leveling policy, total memory consumption and allocation performance.

Another NVM allocator is nvm\_malloc proposed by Schwalb et al.\cite{schwalbnvm},
presenting a general-purpose NVM allocation concept with fast builtin recovery feartures
as well as fine-grained access to NVM in a safe fashion.
Introducing recovery features causes nvm\_malloc incurring higher costs
because of the reserve/activate mechanism and the lack of the maturity as a research prototype mentioned by its authors.
A follow-up comparison to the mature version of nvm\_malloc is subject to future work.

\section{Conclusion and future work}

In this paper, we observe the fact that
no NVM allocators at the time of writing can provide both an accurate wear-leveling policy and a good allocation performance simultaneously.
Based on the observation, we propose an efficient wear-aware NVM allocator, Wamalloc,
which uses a thread-cache memory architecture causing nearly no lock contention in critical path,
and an elaborate hybrid wear-leveling policy to improve lifetime of NVM.
The evaluation results show that...
In our future work, we will continue to optimize the wear-leveling policy based on our byhrid method,
or a log-structure way\cite{rumble2014log} to organize NVM memory.
Performance scalability is also subject to future work.

% For peer review papers, you can put extra information on the cover
% page as needed:
% \ifCLASSOPTIONpeerreview
% \begin{center} \bfseries EDICS Category: 3-BBND \end{center}
% \fi
%
% For peerreview papers, this IEEEtran command inserts a page break and
% creates the second title. It will be ignored for other modes.
\IEEEpeerreviewmaketitle



% use section* for acknowledgement
%\section*{Acknowledgment}

%The authors would like to thank...


% trigger a \newpage just before the given reference
% number - used to balance the columns on the last page
% adjust value as needed - may need to be readjusted if
% the document is modified later
%\IEEEtriggeratref{8}
% The "triggered" command can be changed if desired:
%\IEEEtriggercmd{\enlargethispage{-5in}}

% references section

% can use a bibliography generated by BibTeX as a .bbl file
% BibTeX documentation can be easily obtained at:
% http://www.ctan.org/tex-archive/biblio/bibtex/contrib/doc/
% The IEEEtran BibTeX style support page is at:
% http://www.michaelshell.org/tex/ieeetran/bibtex/
%\bibliographystyle{IEEEtran}
% argument is your BibTeX string definitions and bibliography database(s)
%\bibliography{IEEEabrv,../bib/paper}
%
% <OR> manually copy in the resultant .bbl file
% set second argument of \begin to the number of references
% (used to reserve space for the reference number labels box)

\bibliographystyle{IEEEtran}
\bibliography{IEEE_paper}


% that's all folks
\end{document}


